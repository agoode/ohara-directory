\setuppapersize[letter][letter]

\definefontfamily [office] [serif] [TeX Gyre Termes]
\definefontfamily [office] [sans]  [TeX Gyre Heros]
\definefontfamily [office] [mono]  [TeX Gyre Cursor] [features=none]
\definefontfamily [office] [math]  [TeX Gyre Termes Math]
\setupbodyfont[office, 10pt]

\setupexternalfigures[directory={numbered-forms}]
%\useMPlibrary[dum]

\setuplayout[backspace=0.75in,
             width=7in,
             topspace=0.35in,
             height=10.15in,
             header=0.25in,
             footer=0.25in]

\setuphead[section][indentnext=yes]

\setuppagenumbering[location=]

\startmode[proof]
\setupheadertexts[{\color[red]{DO NOT DISTRIBUTE}}][{\color[red]{PROOFREADER USE ONLY}}][{\color[red]{DO NOT DISTRIBUTE}}][{\color[red]{PROOFREADER USE ONLY}}]
\setupfootertexts[{\color[red]{PROOFREADER USE ONLY}}][{\color[red]{DO NOT DISTRIBUTE}}][{\color[red]{PROOFREADER USE ONLY}}][{\color[red]{DO NOT DISTRIBUTE}}]

\setuppagenumbering[location=bottom]
\definelist[debugno]
\stopmode

\definelist[students]

\setupinteraction[state=start, title={2016 O'Hara Elementary Student Directory}]

\startmode[proof]
\placebookmarks[students]
\placebookmarks[debugno]
\setupinteractionscreen[option=bookmark]
\stopmode

%\setupwhitespace[medium]
\setupinterlinespace[line=2.4ex]

\ss

\startmode[proof]
\title{O'Hara Elementary Directory \color[red]{PROOF DOCUMENT} 2016-10-02}
\stopmode


\starttext

\startmode[proof]
\completecontent
\pagebreak
\stopmode

\startluacode
function splitline(line)
  return string.split(line, '')
end

function readdata(f)
  -- read sqlite 0x1F/0x1E style data (with header)
  local s = string.split(f, '')
  local data = {}
  local header = splitline(s[1])
  table.remove(s, 1)

  -- read lines
  for _,v in ipairs(s) do
    local row = {}
    local l = splitline(v)
    for i,val in ipairs(l) do
      row[header[i]] = string.strip(val)
    end
    -- don't insert empty last row
    if row[header[1]] ~= "" then
      table.insert(data, row)
    end
  end

  return data
end

function unlisted(r)
  if r['Returned?'] == "" then return false end
  return r['Parent/Guardian 1'] == "" and
         r['Parent/Guardian 2'] == "" and
         r['Parent/Guardian 3'] == "" and
         r['Parent/Guardian 4'] == "" and
         r['HOME PHONE 1/2'] == "" and
         r['HOME PHONE 3+4'] == "" and
         r['Str # 1+2'] == "" and
         r['Str # 3+4'] == "" and
         r['St. 1+2'] == "" and
         r['St. 3+4'] == "" and
         r['ZIP 1+2'] == "" and
         r['ZIP 3+4'] == "" and
         r['CELL 1'] == "" and
         r['CELL 2'] == "" and
         r['CELL 3'] == "" and
         r['CELL 4'] == "" and
         r['EMAIL 1'] == "" and
         r['EMAIL 2'] == "" and
         r['EMAIL 3'] == "" and
         r['EMAIL 4'] == ""
end

function issamehomes(a, b)
  a = a or {}
  b = b or {}

  -- no data means no match
  if unlisted(a) or unlisted(b) then return false end

  return a['Parent/Guardian 1'] == b['Parent/Guardian 1'] and
         a['Parent/Guardian 2'] == b['Parent/Guardian 2'] and
         a['Parent/Guardian 3'] == b['Parent/Guardian 3'] and
         a['Parent/Guardian 4'] == b['Parent/Guardian 4'] and
         a['HOME PHONE 1/2'] == b['HOME PHONE 1/2'] and
         a['HOME PHONE 3+4'] == b['HOME PHONE 3+4'] and
         a['Str # 1+2'] == b['Str # 1+2'] and
         a['Str # 3+4'] == b['Str # 3+4'] and
         a['St. 1+2'] == b['St. 1+2'] and
         a['St. 3+4'] == b['St. 3+4'] and
         a['ZIP 1+2'] == b['ZIP 1+2'] and
         a['ZIP 3+4'] == b['ZIP 3+4'] and
         a['CELL 1'] == b['CELL 1'] and
         a['CELL 2'] == b['CELL 2'] and
         a['CELL 3'] == b['CELL 3'] and
         a['CELL 4'] == b['CELL 4'] and
         a['EMAIL 1'] == b['EMAIL 1'] and
         a['EMAIL 2'] == b['EMAIL 2'] and
         a['EMAIL 3'] == b['EMAIL 3'] and
         a['EMAIL 4'] == b['EMAIL 4']
end

function extractname(r)
  local l = {}
  l['first'] = r['FIRST']
  l['last'] = r['LAST']
  l['teacher'] = r['TEACHER']
  l['grade'] = r['GRADE']
  l['no'] = r['No']
  l['extrano'] = r['Extra form']
  l['notes'] = r['Notes']
  return l
end

function printnames(names, proof)
  for i,n in ipairs(names) do
    if not proof then
      context.bookmark({'students'}, string.formatter('%!tex!, %!tex!', n['last'], n['first']))
    end
    context.bold(
      function() context('%!tex!,~%!tex! ', n['last'], n['first'])
                 if proof then context.color({'red'}, function() context(' %d', n['no']) end) end
                 context.hfill()
                 context.unhyphenated(function() context('(%!tex!,~%!tex!)',n['grade'], n['teacher']) end)
      end
    )
    context(true)
  end
end

function printaddr(r, num)
  if r['St. ' .. num] == "" then return end
  local addr = r['Str # ' .. num] .. " " .. r['St. ' .. num] .. ", " .. r['ZIP ' .. num]
  context.indenting()
  context.par()
  context('%!tex!', addr)
  context(true)
end

function printhomephone(r, num)
  printphone(r, num, 'H', 'HOME PHONE')
end

function printcellphone(r, num)
  printphone(r, num, 'C', 'CELL')
end

function printemail(r, num)
  local e = r['EMAIL ' .. num]
  if e == "" then return end
  local emails = string.split(e, ';')
  for k,v in ipairs(emails) do
    context.unhyphenated(function() context('%!tex!', string.strip(v)) end)
    context(true)
  end
end

function printphone(r, num, prefix, key)
  local p = r[key .. ' ' .. num]
  if p == "" then return end
  context('%s: %!tex!', prefix, p)
  context(true)
end

function printparent(r, num, skip, addr)
  local p = r['Parent/Guardian ' .. num]
  if p == "" then return end
  if skip then context.medskip() end
  context.setupindenting({'yes', 'small'})
  context.start()
  context.definedfont({'SansBold at 9pt'})
  context.setupinterlinespace()
  context('%!tex!', p)
  context.stop()
  context(true)
  context.setupindenting({'yes', 'medium'})
  if addr then
    printaddr(r, addr)
    printhomephone(r, addr)
  end
  printcellphone(r, num)
  printemail(r, num)
end

function printrecord(r)
  if r['Returned?'] == "" then return end
  context.setupindenting({'yes', 'small'})
  printhomephone(r, '1/2')
  printaddr(r, '1+2')
  printparent(r, '1', true)
  printparent(r, '2')
  printparent(r, '3', true, '3+4')
  printparent(r, '4')
end

function printdebugimage(img, last, first, notes)
  context.bookmark({'debugno'}, string.formatter('%!tex! %!tex!, %!tex!', img, last, first))
  context.externalfigure({img .. '-small.jpg'}, {width='5in', frame='on'})
  if notes ~= "" then
    context.startlines()
    context('Notes:\r%!tex!', notes:gsub('\n', '\r'))
    context.stoplines()
  end
  context(true)
end

function printentry(names, entry, proof)
  context.startframedtext({width='broad', frame='none'})
  context.startlines()
  printnames(names, proof)
  context.definedfont({'Sans at 9pt'})
  context.setupinterlinespace()
  printrecord(entry)
  context.stoplines()
  context.stopframedtext()
end

function printdirectory(proof)
  if not proof then
    context.startcolumns({n=2})
  end

  local f = readdata(io.loaddata('responses-20160915.list'))
  local prev = nil
  local cur = nil
  local names = {}
  local i = nil
  while true do
    i, cur = next(f, i)
    if prev == nil then
      -- first record
      prev = cur
      table.insert(names, extractname(prev))
      i, cur = next(f, i)
    end

    if cur == nil or not issamehomes(prev, cur) then
      -- flush prev
      if proof then
        for k,v in ipairs(names) do
          printdebugimage(string.format('%0.3d', v['no']), v['last'], v['first'], v['notes'])
          printentry(names, prev, proof)
          context.pagebreak()
          if v['extrano'] ~= "" then
            printdebugimage(string.format('%s', v['extrano']), v['last'], v['first'], v['notes'])
            printentry(names, prev, proof)
            context.pagebreak()
          end
        end
      else
        printentry(names, prev, proof)
      end
      names = {}
      prev = cur
    end

    if prev == nil then break end
    table.insert(names, extractname(cur))
  end

  if not proof then
    context.stopcolumns()
  end
end

function gradetoword(grade)
  if     grade == '0' then return 'KINDERGARTEN'
  elseif grade == '1' then return 'FIRST GRADE'
  elseif grade == '2' then return 'SECOND GRADE'
  elseif grade == '3' then return 'THIRD GRADE'
  elseif grade == '4' then return 'FOURTH GRADE'
  elseif grade == '5' then return 'FIFTH GRADE'
  end
end

function printstudents()
  local f = readdata(io.loaddata('students-20161003.list'))
  local grade = nil
  local teacher = nil
  local inlines = false
  for _, row in ipairs(f) do
    local g = gradetoword(row['GRADE'])
    local t = row['TEACHER']
    local name = row['FIRST'] .. ' ' .. row['LAST']
    commands.writestatus('student', string.formatter('%s %s %s', g, t, name))
    if g ~= grade then
      grade = g
      if inlines then
        context.stoplines()
        context.stopcolumns()
        context.vfill()
        inlines = false
      end
      context.pagebreak()
      context.startframedtext({frame='off', width='broad', align='middle', bodyfont='18pt', style='bold'})
      context(g)
      context.stopframedtext()
    end
    if t ~= teacher then
      teacher = t
      if inlines then
        context.stoplines()
        context.stopcolumns()
        context.vfill()
        inlines = false
      end
      context.start()
      context.hfill()
      context.bold()
      context(t)
      context.hfill()
      context.stop()
      context(true)
      context.medskip()
      context.startcolumns({n=4, balance='yes'})
      context.startlines()
      inlines = true
    end
    context.unhyphenated(name)
    context(true)
  end
  if inlines then
    context.stoplines()
    context.stopcolumns()
    context.vfill()
    inlines = false
  end
end

local samplenames = {{first='First', last='Last', teacher='TEACHER', grade='1'},
                     {first='Second', last='Last', teacher='PROFESSOR', grade='5'}}
local sampleentry = {['Parent/Guardian 1'] = 'Parent/Guardian 1',
                     ['Parent/Guardian 2'] = 'Parent/Guardian 2',
                     ['Parent/Guardian 3'] = 'Parent/Guardian 3',
                     ['Parent/Guardian 4'] = 'Parent/Guardian 4',
                     ['HOME PHONE 1/2'] = '121-212-1212',
                     ['HOME PHONE 3+4'] = '343-434-3434',
                     ['Str # 1+2'] = '12',
                     ['Str # 3+4'] = '34',
                     ['St. 1+2'] = 'First St.',
                     ['St. 3+4'] = 'Third St.',
                     ['ZIP 1+2'] = '11111',
                     ['ZIP 3+4'] = '33333',
                     ['CELL 1'] = '111-111-1111',
                     ['CELL 2'] = '222-222-2222',
                     ['CELL 3'] = '333-333-3333',
                     ['CELL 4'] = '444-444-4444',
                     ['EMAIL 1'] = 'first@example.com; first+alt@example.com',
                     ['EMAIL 2'] = 'second@example.com',
                     ['EMAIL 3'] = 'third@example.com',
                     ['EMAIL 4'] = 'fourth@example.com'}

if (tex.modes["proof"]) then
  context.section('Sample format')
  context.startcolumns({n=2})
  printentry(samplenames, sampleentry)
  context.stopcolumns()
  context.pagebreak()

  context.section('Layout draft')
  context('Starts on next page.')
  context.pagebreak()
  printdirectory(false)
  context.pagebreak()

  context.section('Proofreading')
  context('Starts on next page.')
  context.pagebreak()
  printdirectory(true)

  context.section('Student list by teacher')
  context('Starts on next page.')
  context.pagebreak()
  printstudents()
else
  printdirectory(false)
  printstudents()
end

\stopluacode
\stoptext
