\setuppapersize[letter][letter]

\definefontfamily [office] [serif] [TeX Gyre Termes]
\definefontfamily [office] [sans]  [TeX Gyre Heros]
\definefontfamily [office] [mono]  [TeX Gyre Cursor] [features=none]
\definefontfamily [office] [math]  [TeX Gyre Termes Math]
\setupbodyfont[office, 10pt]

\startmode[tiny]
\setupbodyfont[office, 2pt]
\stopmode

\setupexternalfigures[directory={numbered-forms}]
%\useMPlibrary[dum]

\setuplayout[backspace=0.75in,
             width=7in,
             topspace=0.35in,
             height=10.15in,
             header=0.25in,
             footer=0.25in]

\startmode[tiny]
\setuplayout[backspace=0.25in,
             width=8in,
             topspace=0.25in,
             height=10.50in,
             header=0in,
             footer=0in]
\stopmode

\setuphead[section][indentnext=yes]

\setuppagenumbering[location=]

\startmode[proof]
\setupheadertexts[{\color[red]{DO NOT DISTRIBUTE}}][{\color[red]{PROOFREADER USE ONLY}}][{\color[red]{DO NOT DISTRIBUTE}}][{\color[red]{PROOFREADER USE ONLY}}]
\setupfootertexts[{\color[red]{PROOFREADER USE ONLY}}][{\color[red]{DO NOT DISTRIBUTE}}][{\color[red]{PROOFREADER USE ONLY}}][{\color[red]{DO NOT DISTRIBUTE}}]

\setuppagenumbering[location=bottom]
\definelist[debugno]
\stopmode

\definelist[students]

\setupinteraction[state=start, title={2019 O'Hara Elementary Student Directory}]

\startmode[proof]
\placebookmarks[students]
\placebookmarks[debugno]
\setupinteractionscreen[option=bookmark]
\stopmode

%\setupwhitespace[medium]
\setupinterlinespace[line=2.4ex]

\ss

\startmode[proof]
\title{O'Hara Elementary Directory \color[red]{PROOF DOCUMENT} 2019-10-10}
\stopmode


\starttext

\startmode[proof]
\completecontent
\pagebreak
\stopmode

\startluacode
function splitline(line)
  return string.split(line, '')
end

function readdata(f)
  -- read sqlite 0x1F/0x1E style data (with header)
  local s = string.split(f, '')
  local data = {}
  local header = splitline(s[1])
  table.remove(s, 1)

  -- read lines
  for _,v in ipairs(s) do
    local row = {}
    local l = splitline(v)
    for i,val in ipairs(l) do
      row[header[i]] = string.strip(val)
    end
    -- don't insert empty last row
    if row[header[1]] ~= "" then
      table.insert(data, row)
    end
  end

  return data
end

function issamehomes(a, b)
  a = a or {}
  b = b or {}

  return (a['cname1'] == b['cname1'] and
          a['cname2'] == b['cname2'])
         or
         (a['cphone1'] == b['cphone1'] and
          a['caddress1'] == b['caddress1'] and
          a['cphone2'] == b['cphone2'] and
          a['caddress2'] == b['caddress2'])
         or
         (a['cemail1'] == b['cemail1'] and
          a['cemail2'] == b['cemail2'])

end

function extractname(r)
  local l = {}
  l['first'] = r['first']
  l['last'] = r['last']
  l['teacher'] = r['teacher']
  l['grade'] = r['grade']
  return l
end

function printnames(names, proof)
  for i,n in ipairs(names) do
    s = string.formatter('%!tex!, %!tex!', n['last'], n['first'])
    context.bookmark({'students'}, s)
    context.bold(
      function() s = string.formatter('%!tex!,~%!tex!', n['last'], n['first'])
                 tex.sprint(s)
                 context.hfill()
                 local grade = n['grade']
                 if grade == '0' then grade = 'K' end
                 context.unhyphenated(function() context('(%!tex!,~%!tex!)', grade, n['teacher']) end)
      end
    )
    context(true)
  end
end

function printaddr(r, num)
  local a = r['caddress' .. num]
  if a == "" then return end
  context.indenting()
  context.par()
  context('%!tex!', a)
  context(true)
end

function printemail(r, num)
  local e = r['cemail' .. num]
  if e == "" then return end
  local emails = string.split(e, ';')
  for k,v in ipairs(emails) do
    context.unhyphenated(function() context('%!tex!', string.strip(v)) end)
    context(true)
  end
end

function printphone(r, num)
  local p = r['cphone' .. num]
  if p == "" then return end
  context('%!tex!', p)
  context(true)
end

function printparent(r, num)
  local p = r['cname' .. num]
  if p ~= "" then
    context.setupindenting({'yes', 'small'})
    context.start()
    if tex.modes["tiny"] then
      context.definedfont({'SansBold at 2pt'})
    else
      context.definedfont({'SansBold at 9pt'})
    end
    context.setupinterlinespace()
    context('%!tex!', p)
    context.stop()
    context(true)
  end
  context.setupindenting({'yes', 'medium'})
  printaddr(r, num)
  printphone(r, num)
  printemail(r, num)
end

function printrecord(r)
  context.setupindenting({'yes', 'small'})
  printparent(r, '1')
  printparent(r, '2')
end

function printentry(names, entry, proof)
  context.startframedtext({width='broad', frame='none'})
  context.startlines()
  printnames(names, proof)
  if tex.modes["tiny"] then
    context.definedfont({'Sans at 2pt'})
  else
    context.definedfont({'Sans at 9pt'})
  end
  context.setupinterlinespace()
  printrecord(entry)
  context.stoplines()
  context.stopframedtext()
end

function cleanentry(e)
  if e == nil then return end
  for i = 1, 2 do
    if not string.find(e['caddress' .. i], '%d') then e['caddress' .. i] = '' end
    if not string.find(e['cemail' .. i], '@') then e['cemail' .. i] = '' end
    if not string.find(e['cphone' .. i], '%d') then e['cphone' .. i] = '' end
  end
end

function mergeentry(old, new)
  if old == nil or new == nil then return end
  for i = 1, 2 do
    a = 'caddress' .. i
    e = 'cemail' .. i
    p = 'cphone' .. i
    if old[a] == "" then old[a] = new[a] end
    if old[e] == "" then old[e] = new[e] end
    if old[p] == "" then old[p] = new[p] end
  end
end

function printdirectory(proof)
  if tex.modes["tiny"] then
    context.startcolumns({n=9})
  else
    context.startcolumns({n=2})
  end

  local f = readdata(io.loaddata('export-20191010.list'))
  local entry = nil
  local nextentry = nil
  local names = {}
  local i = nil
  local familycount = 0
  while true do
    i, nextentry = next(f, i)
    cleanentry(nextentry)
    if entry == nil then
      -- first record
      entry = nextentry
      table.insert(names, extractname(entry))
      i, nextentry = next(f, i)
    end

    if nextentry == nil or not issamehomes(entry, nextentry) then
      -- flush entry
      familycount = familycount + 1
      commands.writestatus('family', string.formatter('%d', familycount))
      printentry(names, entry, proof)
      names = {}
      entry = nextentry
    else
      mergeentry(entry, nextentry)
    end

    if entry == nil then break end
    table.insert(names, extractname(nextentry))
  end
  context.stopcolumns()
end

function gradetoword(grade)
  if     grade == '0' then return 'KINDERGARTEN'
  elseif grade == '1' then return 'FIRST GRADE'
  elseif grade == '2' then return 'SECOND GRADE'
  elseif grade == '3' then return 'THIRD GRADE'
  elseif grade == '4' then return 'FOURTH GRADE'
  elseif grade == '5' then return 'FIFTH GRADE'
  end
end

function printstudents()
  local f = readdata(io.loaddata('students-20191010.list'))
  local grade = nil
  local teacher = nil
  local inlines = false
  for i, row in ipairs(f) do
    local g = gradetoword(row['grade'])
    local t = row['teacher']
    local name = row['first'] .. ' ' .. row['last']
    commands.writestatus('student', string.formatter('%d %s %s %s', i, g, t, name))
    if g ~= grade then
      grade = g
      if inlines then
        context.stoplines()
        context.stopcolumns()
        context.vfill()
        inlines = false
      end
      context.pagebreak()
      context.startframedtext({frame='off', width='broad', align='middle', bodyfont='18pt', style='bold'})
      context(g)
      context.stopframedtext()
    end
    if t ~= teacher then
      teacher = t
      if inlines then
        context.stoplines()
        context.stopcolumns()
        context.vfill()
        inlines = false
      end
      context.start()
      context.hfill()
      context.bold()
      context(t)
      context.hfill()
      context.stop()
      context(true)
      context.medskip()
      context.startcolumns({n=4, balance='yes'})
      context.startlines()
      inlines = true
    end
    context.unhyphenated(name)
    context(true)
  end
  if inlines then
    context.stoplines()
    context.stopcolumns()
    context.vfill()
    inlines = false
  end
end

function emptyentry()
  local entry = {cname1 = '',
                 cphone1 = '',
                 cemail1 = '',
                 caddress1 = '',
                 cname2 = '',
                 cphone2 = '',
                 cemail2 = '',
                 caddress2 = ''}
  return entry
end

local samplenames = {{first='First', last='Last', teacher='TEACHER', grade='1'},
                     {first='Second', last='Last', teacher='PROFESSOR', grade='5'}}
local sampleentry = {cname1 = 'cname1',
                     cphone1 = '021-212-1212',
                     cemail1 = 'first@example.com; first+alt@example.com',
                     caddress1 = '123 First St. City ZZ 33333',
                     cname2 = 'cname2',
                     cphone2 = '121-212-1212',
                     cemail2 = 'second@example.com',
                     caddress2 = '456 Second St. City ZZ 44444'}

if tex.modes["proof"] then
  context.section('Sample format')
  context.startcolumns({n=2})

  printentry(samplenames, sampleentry)

  local justemailentry = emptyentry()
  justemailentry['cemail1'] = 'just@example.com'
  printentry({{first='Just', last='Email', teacher='TEACHER', grade='0'}},
             justemailentry)

  local justemailentry2 = emptyentry()
  justemailentry2['cemail2'] = 'just2@example.com'
  printentry({{first='Another', last='Email', teacher='TEACHER', grade='0'}},
             justemailentry2)

  context.stopcolumns()
  context.pagebreak()

  context.section('Layout draft')
  context('Starts on next page.')
  context.pagebreak()
  printdirectory(true)
  context.pagebreak()

  context.section('Student list by teacher')
  context('Starts on next page.')
  context.pagebreak()
  printstudents()
else
  printdirectory(false)
  if not tex.modes["tiny"] then
    printstudents()
  end
end

\stopluacode
\stoptext
