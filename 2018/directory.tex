\setuppapersize[letter][letter]

\definefontfamily [office] [serif] [TeX Gyre Termes]
\definefontfamily [office] [sans]  [TeX Gyre Heros]
\definefontfamily [office] [mono]  [TeX Gyre Cursor] [features=none]
\definefontfamily [office] [math]  [TeX Gyre Termes Math]
\setupbodyfont[office, 10pt]

\startmode[tiny]
\setupbodyfont[office, 2pt]
\stopmode

\setupexternalfigures[directory={numbered-forms}]
%\useMPlibrary[dum]

\setuplayout[backspace=0.75in,
             width=7in,
             topspace=0.35in,
             height=10.15in,
             header=0.25in,
             footer=0.25in]

\startmode[tiny]
\setuplayout[backspace=0.25in,
             width=8in,
             topspace=0.25in,
             height=10.50in,
             header=0in,
             footer=0in]
\stopmode

\setuphead[section][indentnext=yes]

\setuppagenumbering[location=]

\startmode[proof]
\setupheadertexts[{\color[red]{DO NOT DISTRIBUTE}}][{\color[red]{PROOFREADER USE ONLY}}][{\color[red]{DO NOT DISTRIBUTE}}][{\color[red]{PROOFREADER USE ONLY}}]
\setupfootertexts[{\color[red]{PROOFREADER USE ONLY}}][{\color[red]{DO NOT DISTRIBUTE}}][{\color[red]{PROOFREADER USE ONLY}}][{\color[red]{DO NOT DISTRIBUTE}}]

\setuppagenumbering[location=bottom]
\definelist[debugno]
\stopmode

\definelist[students]

\setupinteraction[state=start, title={2018 O'Hara Elementary Student Directory}]

\startmode[proof]
\placebookmarks[students]
\placebookmarks[debugno]
\setupinteractionscreen[option=bookmark]
\stopmode

%\setupwhitespace[medium]
\setupinterlinespace[line=2.4ex]

\ss

\startmode[proof]
\title{O'Hara Elementary Directory \color[red]{PROOF DOCUMENT} 2018-10-12}
\stopmode


\starttext

\startmode[proof]
\completecontent
\pagebreak
\stopmode

\startluacode
function splitline(line)
  return string.split(line, '')
end

function readdata(f)
  -- read sqlite 0x1F/0x1E style data (with header)
  local s = string.split(f, '')
  local data = {}
  local header = splitline(s[1])
  table.remove(s, 1)

  -- read lines
  for _,v in ipairs(s) do
    local row = {}
    local l = splitline(v)
    for i,val in ipairs(l) do
      row[header[i]] = string.strip(val)
    end
    -- don't insert empty last row
    if row[header[1]] ~= "" then
      table.insert(data, row)
    end
  end

  return data
end

function issamehomes(a, b)
  a = a or {}
  b = b or {}

  return a['apt'] == b['apt'] and
         a['street'] == b['street'] and
         a['city'] == b['city'] and
         a['state'] == b['state'] and
         a['zip'] == b['zip'] and
         a['home0'] == b['home0'] and
         a['pg1'] == b['pg1'] and
         a['home1'] == b['home1'] and
         a['cell1'] == b['cell1'] and
         a['email1'] == b['email1'] and
         a['pg2'] == b['pg2'] and
         a['home2'] == b['home2'] and
         a['cell2'] == b['cell2'] and
         a['email2'] == b['email2'] and
         a['pg3'] == b['pg3'] and
         a['home3'] == b['home3'] and
         a['cell3'] == b['cell3'] and
         a['email4'] == b['email4']
end

function extractname(r)
  local l = {}
  l['no'] = r['no']
  l['first'] = r['first']
  l['last'] = r['last']
  l['teacher'] = r['teacher']
  l['grade'] = r['grade']
  return l
end

function printnames(names, proof)
  for i,n in ipairs(names) do
    s = string.formatter('%!tex!, %!tex!', n['last'], n['first'])
    context.bookmark({'students'}, s)
    context.bold(
      function() s = string.formatter('%!tex!,~%!tex!', n['last'], n['first'])
                 tex.sprint(s)
                 if proof then context.color({'red'}, function() context(' row %d', n['no']) end) end
                 context.hfill()
                 context.unhyphenated(function() context('(%!tex!,~%!tex!)',n['grade'], n['teacher']) end)
      end
    )
    context(true)
  end
end

function printaddr(r)
  if r['street'] == "" then return end
  local street = r['street']
  local apt = r['apt']
  if string.find(apt, '^%d+$') ~= nil then apt = '# ' .. apt end
  if apt ~= "" then street = street .. ' ' .. apt end
  local addr = string.formatter('%s, %s %s %s', street, r['city'], r['state'], r['zip'])
  context.indenting()
  context.par()
  context('%!tex!', addr)
  context(true)
end

function printhomephone(r, num)
  printphone(r, num, 'H', 'home')
end

function printcellphone(r, num)
  printphone(r, num, 'C', 'cell')
end

function printemail(r, num)
  local e = r['email' .. num]
  if e == "" then return end
  local emails = string.split(e, ';')
  for k,v in ipairs(emails) do
    context.unhyphenated(function() context('%!tex!', string.strip(v)) end)
    context(true)
  end
end

function printphone(r, num, prefix, key)
  local p = r[key .. num]
  if p == "" then return end
  context('%s: %!tex!', prefix, p)
  context(true)
end

function printparent(r, num, skip)
  local p = r['pg' .. num]
  if skip then context.medskip() end
  if p ~= "" then
    context.setupindenting({'yes', 'small'})
    context.start()
    if tex.modes["tiny"] then
      context.definedfont({'SansBold at 2pt'})
    else
      context.definedfont({'SansBold at 9pt'})
    end
    context.setupinterlinespace()
    context('%!tex!', p)
    context.stop()
    context(true)
  end
  context.setupindenting({'yes', 'medium'})
  printhomephone(r, num)
  printcellphone(r, num)
  printemail(r, num)
end

function printrecord(r)
  context.setupindenting({'yes', 'small'})
  printaddr(r)
  printhomephone(r, '0')
  printparent(r, '1', true)
  printparent(r, '2')
  printparent(r, '3')
end

function printentry(names, entry, proof)
  context.startframedtext({width='broad', frame='none'})
  context.startlines()
  printnames(names, proof)
  if tex.modes["tiny"] then
    context.definedfont({'Sans at 2pt'})
  else
    context.definedfont({'Sans at 9pt'})
  end
  context.setupinterlinespace()
  printrecord(entry)
  context.stoplines()
  context.stopframedtext()
end

function printdirectory(proof)
  if tex.modes["tiny"] then
    context.startcolumns({n=9})
  else
    context.startcolumns({n=2})
  end

  local f = readdata(io.loaddata('export-20181012.list'))
  local prev = nil
  local cur = nil
  local names = {}
  local i = nil
  local familycount = 0
  while true do
    i, cur = next(f, i)
    if prev == nil then
      -- first record
      prev = cur
      table.insert(names, extractname(prev))
      i, cur = next(f, i)
    end

    if cur == nil or not issamehomes(prev, cur) then
      -- flush prev
      familycount = familycount + 1
      commands.writestatus('family', string.formatter('%d', familycount))
      printentry(names, prev, proof)
      names = {}
      prev = cur
    end

    if prev == nil then break end
    table.insert(names, extractname(cur))
  end
  context.stopcolumns()
end

function gradetoword(grade)
  if     grade == '0' then return 'KINDERGARTEN'
  elseif grade == '1' then return 'FIRST GRADE'
  elseif grade == '2' then return 'SECOND GRADE'
  elseif grade == '3' then return 'THIRD GRADE'
  elseif grade == '4' then return 'FOURTH GRADE'
  elseif grade == '5' then return 'FIFTH GRADE'
  end
end

function printstudents()
  local f = readdata(io.loaddata('students-20181012.list'))
  local grade = nil
  local teacher = nil
  local inlines = false
  for i, row in ipairs(f) do
    local g = gradetoword(row['grade'])
    local t = row['teacher']
    local name = row['first'] .. ' ' .. row['last']
    commands.writestatus('student', string.formatter('%d %s %s %s', i, g, t, name))
    if g ~= grade then
      grade = g
      if inlines then
        context.stoplines()
        context.stopcolumns()
        context.vfill()
        inlines = false
      end
      context.pagebreak()
      context.startframedtext({frame='off', width='broad', align='middle', bodyfont='18pt', style='bold'})
      context(g)
      context.stopframedtext()
    end
    if t ~= teacher then
      teacher = t
      if inlines then
        context.stoplines()
        context.stopcolumns()
        context.vfill()
        inlines = false
      end
      context.start()
      context.hfill()
      context.bold()
      context(t)
      context.hfill()
      context.stop()
      context(true)
      context.medskip()
      context.startcolumns({n=4, balance='yes'})
      context.startlines()
      inlines = true
    end
    context.unhyphenated(name)
    context(true)
  end
  if inlines then
    context.stoplines()
    context.stopcolumns()
    context.vfill()
    inlines = false
  end
end

function emptyentry()
  local entry = {['pg1'] = '',
                 ['pg2'] = '',
                 ['pg3'] = '',
                 ['apt'] = '',
                 ['street'] = '',
                 ['city'] = '',
                 ['state'] = '',
                 ['zip'] = '',
                 ['home0'] = '',
                 ['home1'] = '',
                 ['home2'] = '',
                 ['home3'] = '',
                 ['cell1'] = '',
                 ['cell2'] = '',
                 ['cell3'] = '',
                 ['email1'] = '',
                 ['email2'] = '',
                 ['email3'] = ''}
  return entry
end

local samplenames = {{first='First', last='Last', teacher='TEACHER', grade='1'},
                     {first='Second', last='Last', teacher='PROFESSOR', grade='5'}}
local sampleentry = {['pg1'] = 'pg1',
                     ['pg2'] = 'pg2',
                     ['pg3'] = 'pg3',
                     ['home0'] = '021-212-1212',
                     ['home1'] = '121-212-1212',
                     ['home2'] = '221-212-1212',
                     ['home3'] = '321-212-1212',
                     ['apt'] = '5',
                     ['street'] = '123 First St.',
                     ['city'] = 'City',
                     ['state'] = 'ZZ',
                     ['zip'] = '33333',
                     ['cell1'] = '111-111-1111',
                     ['cell2'] = '222-222-2222',
                     ['cell3'] = '333-333-3333',
                     ['email1'] = 'first@example.com; first+alt@example.com',
                     ['email2'] = 'second@example.com',
                     ['email3'] = 'third@example.com'}

if tex.modes["proof"] then
  context.section('Sample format')
  context.startcolumns({n=2})

  printentry(samplenames, sampleentry)

  local justemailentry = emptyentry()
  justemailentry['email1'] = 'just@example.com'
  printentry({{first='Just', last='Email', teacher='TEACHER', grade='K'}},
             justemailentry)

  local justemailentry2 = emptyentry()
  justemailentry2['email2'] = 'just2@example.com'
  printentry({{first='Another', last='Email', teacher='TEACHER', grade='K'}},
             justemailentry2)

  context.stopcolumns()
  context.pagebreak()

  context.section('Layout draft')
  context('Starts on next page. Row numbers from original input added for reference.')
  context.pagebreak()
  printdirectory(true)
  context.pagebreak()

  context.section('Student list by teacher')
  context('Starts on next page.')
  context.pagebreak()
  printstudents()
else
  printdirectory(false)
  if not tex.modes["tiny"] then
    printstudents()
  end
end

\stopluacode
\stoptext
